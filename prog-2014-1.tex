% ------------------------------------------------------------------------------
% Unravelling the mysteries of ergodic processes
%
% 2014-02-24T15:18:01
% ------------------------------------------------------------------------------
\documentclass[11pt]{article}
\usepackage[a4paper,margin=2.5cm,top=2.0cm,bottom=2.0cm]{geometry}
\usepackage{newtxtext}
\usepackage{libertine}
\usepackage[libertine,cmintegrals,cmbraces]{newtxmath}
\usepackage[T1]{fontenc}
\usepackage[small,compact]{titlesec}
\usepackage[brazil]{babel}
\usepackage[square]{natbib}
\usepackage{datenumber}
\usepackage{longtable}
\usepackage{amsmath}
\usepackage{tabu}
\usepackage{booktabs}
\usepackage{url}
\usepackage{textcomp}
\usepackage{fixltx2e}
\newcommand{\ira}{\textsuperscript{\underline{a}}}
\newcommand{\iro}{\textsuperscript{\underline{o}}}
\newcommand{\iras}{\textsuperscript{\underline{as}}}
\newcommand{\iros}{\textsuperscript{\underline{os}}}
\newcommand{\xira}{$^{\underline{\rm a}}$}
\newcommand{\xiro}{$^{\underline{\rm o}}$}
\newcommand{\sub}[1]{\textsubscript{#1}}
\newcommand{\super}[1]{\textsuperscript{#1}}
\newcommand{\eg}{{\it e.g.}}%
\newcommand{\etal}{{\it et~al}.} %
\newcommand{\etals}{{\it et~al}.'s}%
\newcommand{\ie}{{\it i.e.}}%
\newcommand{\quotes}[1]{``#1''}%

\newcommand{\fst}{\textsuperscript{\underline{st}}}
\newcommand{\snd}{\textsuperscript{\underline{nd}}}
\newcommand{\trd}{\textsuperscript{\underline{rd}}}
\newcommand{\fth}{\textsuperscript{\underline{th}}}

\newcommand{\thistt}{%
\ttfamily
\fontdimen2\font=0.4em% interword space
\fontdimen3\font=0.2em% interword stretch
\fontdimen4\font=0.1em% interword shrink
\fontdimen7\font=0.1em% extra space
\hyphenchar\font=`\-% to allow hyphenation
\catcode`\^=11
\catcode`\_=11
\catcode`\$=11
}


% ------------------------------------------------------------------------------
% contadores
% ------------------------------------------------------------------------------
\newcounter{aula}
\setcounter{aula}{0}
\newcounter{inweek}
\setcounter{inweek}{0}
\newcounter{semdi}
% -------------------------------------------------------------------------------
% algoritmo para calcular automaticamente a data da pr�xima aula
% -------------------------------------------------------------------------------
\newcommand{\mplus}{
   \ifcase\value{inweek}
      \addtocounter{datenumber}{5}\setdatebynumber{\thedatenumber} % 4a->2a
   \or
      \addtocounter{datenumber}{2}\setdatebynumber{\thedatenumber} % 2a->4a
   \fi
   \stepcounter{inweek}
   \ifnum\value{inweek}>1
      \setcounter{inweek}{0}
   \fi
}
% --------------------------------------------------------------------------------
% pr�xima aula
% --------------------------------------------------------------------------------
\newcommand{\maisau}{
   \stepcounter{aula}\arabic{aula}\mplus
}
% --------------------------------------------------------------------------------
% 2a, 4a, 6a (n�o funcionaria nos fins de semana)
% --------------------------------------------------------------------------------
\newcommand{\dsem}{%
   \setcounter{semdi}{\value{datedayname}}
   \addtocounter{semdi}{1}
   \arabic{semdi}\ira%
}
% --------------------------------------------------------------------------------
% imprime o m�s e o dia
% --------------------------------------------------------------------------------
\newcommand{\mesdi}{\arabic{dateday}/\arabic{datemonth}}
% --------------------------------------------------------------------------------
% faz um par�grafo na coluna da tabela
% --------------------------------------------------------------------------------
\newcommand{\boxit}[1]{\parbox[t]{10cm}{#1}}
\newcommand{\boxut}[1]{\parbox[t]{4cm}{#1}}


\renewcommand{\ttdefault}{cmtt}% default cmtt


\begin{document}
\setdatenumber{2014}{2}{12} % 4a feira, tr�s dias antes do in�cio das aulas

\title{TEA775-T�picos Especiais em Engenharia Ambiental\\
PPGEA, 1/2014}
\author{Prof. Nelson Lu�s Dias (Lemma, Centro Polit�cnico, 3320-2025)\\
\texttt{nldias@ufpr.br}}
\date{\today}

\maketitle

\section*{Ensalamento e hor�rio}

2\iras\ e  4\iras\, Sala 1 CESEC, 09:30--11:10.

\section*{Objetivos Did�ticos}



\section*{Unidades Did�ticas}


{\tabulinesep=1.5pt
\noindent\begin{tabu}[t]{XX[15]}
\toprule
1 & Introdu��o ao curso\\
2 & Vari�veis aleat�rias\\
\bottomrule
\end{tabu}}





\newpage





\section*{Programa}

\textbf{Aten��o:} Este � um planejamento.  O conte�do individual das aulas
poder� variar de acordo com o andamento da disciplina.


{\fontsize{8pt}{8pt}\selectfont
\tabulinesep=1.5pt
\noindent\begin{tabu}[t]{XXXXX[15]X[5]}
\toprule
Aula & UD & Dia & Data & Conte�do & Progresso \\
\midrule
\maisau & 1 & \dsem & \mesdi & Apresenta��o do curso. &  \\
\maisau & 1 & \dsem & \mesdi & Vari�veis aleat�rias: aditividade finita.      &  \\ 
\maisau & 1 & \dsem & \mesdi & Vari�veis aleat�rias: aditividade infinita.    &  \\
\maisau & 1 & \dsem & \mesdi & O que � uma �lgebra-sigma?    &  \\
\mplus  &   & \dsem & \mesdi & \textbf{Feriado: Carnaval}                               &  \\
\mplus  &   & \dsem & \mesdi & \textbf{Feriado: Carnaval}                               &  \\
\bottomrule
\end{tabu}}










\nocite{birkhoff-proof-ergodic}
\nocite{breuer.petruccione--burgers.model}
\nocite{einstein--uber.molekularkinetischen}
\nocite{einstein--movement}
\nocite{kolmogorov--foundations}
\nocite{langevin--brownien}
\nocite{lemons.gythiel--langevin.brownian}
\nocite{reynolds--b-dynamical}
\nocite{james--probabilidade.intermediario}
\nocite{rosenthal--first.look}
\nocite{todorovic--introduction.stochastic.processes}




\bibliography{all}
\bibliographystyle{belllike}


\end{document}

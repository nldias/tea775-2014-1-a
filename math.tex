\usepackage{amsmath}
\usepackage{amssymb}
\usepackage{textcomp}
\usepackage{stmaryrd}
%\usepackage{bm}
% 
% package relsize: needed only if you use \mathlarger, \mathsmaller!!!
%\usepackage{relsize}
%
\def\Xint#1{\mathchoice
{\XXint\displaystyle\textstyle{#1}}%
{\XXint\textstyle\scriptstyle{#1}}%
{\XXint\scriptstyle\scriptscriptstyle{#1}}%
{\XXint\scriptscriptstyle\scriptscriptstyle{#1}}%
\!\int}
\def\XXint#1#2#3{{\setbox0=\hbox{$#1{#2#3}{\int}$}
\vcenter{\hbox{$#2#3$}}\kern-.5\wd0}}
\def\ddashint{\Xint=}
\def\dashint{\Xint-}

% ---------------------------------------------------------------------------
% use Euler Fonts for my own devious ends
% ---------------------------------------------------------------------------
\DeclareMathAlphabet\EuScript{U}{eus}{m}{n}
\SetMathAlphabet\EuScript{bold}{U}{eus}{b}{n}


\newcommand{\bbxfamily}{\fontencoding{U}\fontfamily{bbold}\selectfont}
\newcommand{\textbbx}[1]{{\bbxfamily#1}}
\DeclareMathAlphabet{\mathbbx}{U}{bbold}{m}{n}
\DeclareMathAlphabet{\mathpzc}{OT1}{pzc}{m}{it}
\newcommand{\zapf}{\fontencoding{OT1}\fontfamily{pzc}\fontseries{m}\fontshape{it}\fontsize{16}{16pt}\selectfont}
% ---------------------------------------------------------------------------
% math.tex was originally developed as a set of TeX macros.  Here, it
% is being modified for LaTeX use, so you wil find a mix of TeX and
% LaTeX commands (sorry...).  At any rate, it provides some useful
% macros.  Take a look
%
% Nelson Lu\'{\i}s Dias
% 15 apr 1994
% 27 sep 1996
% 27 sep 2002
% ---------------------------------------------------------------------------
%\newcounter{ItemNumber} %
% ---------------------------------------------------------------------------
% the fourier-transform of long expressions is typed as a wide hat after
% the expression
% ---------------------------------------------------------------------------
\newcommand{\aft}{\widehat{\phantom{u}}}
% ---------------------------------------------------------------------------
% macros to simplify math typing
% ---------------------------------------------------------------------------
\DeclareMathOperator{\sen}{sen}%
\DeclareMathOperator{\senh}{senh}%
\DeclareMathOperator{\tg}{tg}%
\DeclareMathOperator{\tgh}{tgh}%
\DeclareMathOperator{\cotgh}{cotgh}%
\DeclareMathOperator{\snl}{snl}
\DeclareMathOperator{\arcsen}{arcsen}%
\DeclareMathOperator{\erf}{erf}%
\DeclareMathOperator{\erfc}{erfc}%
\DeclareMathOperator{\haversin}{haversin}%
\DeclareMathOperator{\arcsenh}{arcsenh}%
\DeclareMathOperator{\arctg}{arctg}%
\DeclareMathOperator{\arctgyx}{arctg2}%
\DeclareMathOperator{\cis}{cis}%
\newcommand{\arctgh}{{\rm arctgh}}%
\newcommand{\blob}{\; \mbox{\rule{4pt}{4pt}}}%
\newcommand{\bfvec}[1]{\mathord{\mbox{\bf #1}}}%
\newcommand{\cotg}{{\rm cotg\,}}%
\newcommand{\cross}{\vet{\times}}
\newcommand{\deriva}[2]{\frac{d{#1}}{d{#2}}}%
\newcommand{\mderiva}[2]{\frac{\md{#1}}{\md{#2}}}%
\def\dbar{{\raisebox{-0.75pt}{$\mathchar'26$}\mkern-14.5mu\mathrm{d}}} 
\newcommand{\dimof}[1]{\left\llbracket #1 \right\rrbracket}
\DeclareMathOperator{\Cov}{Cov}%
\DeclareMathOperator{\Var}{Var}%
\DeclareMathOperator{\MSE}{MSE}
\DeclareMathOperator{\RMSE}{RMSE}
\DeclareMathOperator{\EMQ}{EMQ}%
\DeclareMathOperator{\REMQ}{REMQ}
\DeclareMathOperator{\Exv}{E}%
\DeclareMathOperator{\Co}{Co}
\DeclareMathOperator{\Qu}{Qu}
\DeclareMathOperator{\Rea}{Re}
\DeclareMathOperator{\Img}{Im}
% ----------------------------------------------------------------------------
% \derder must be considered obsolete, but is kept for backward compatibility;
% \dern is preferable
% ----------------------------------------------------------------------------
\newcommand{\derder}[3]{%
   \frac{d^{#3}{#1}}
        {d{#2}^{#3}}
}%
\newcommand{\dern}[3]{%
   \frac{d^{#3}{#1}}
        {d{#2}^{#3}}
}%
\newcommand{\mdern}[3]{%
   \frac{\md^{#3}{#1}}
        {\md{#2}^{#3}}
}%
\newcommand{\dmat}[1]{\frac{ D{#1}}{Dt}}%
\newcommand{\dmatbar}[1]{\frac{ \overline{D}{#1}}{Dt}}%
\newcommand{\fof}[1]{\mathord{\mbox{\large\it ff}\,{(#1)}}}%
\newcommand{\ifonlyif}{{\; \Leftrightarrow \;}}%
\providecommand{\implies}{\; \Rightarrow \;}%
\newcommand{\then}{\; \Rightarrow \;}%
\newcommand{\parder}[2]{ {\frac{\partial {#1}}{\partial{#2}} }}%
\newcommand{\thermopar}[3]{{\left(\frac{\partial {#1}}{\partial{#2}}\right)_{#3} }}%

\newcommand{\parpar}[3]{\frac{\partial^2 {#1}}{\partial{#2} \partial{#3}}}%
\newcommand{\parparpar}[4]{\frac{\partial^3 {#1}}{\partial{#2} \partial{#3}\partial{#4}}}%
\newcommand{\parn}[3]{%
    \frac{\partial^{#3} {#1}}%
         {\partial {#2}^{#3}}
}%
\newcommand{\parpow}[2]{ {\left( #1 \right) }^{#2} }%
\newcommand{\such}{\;|\;}%
\newcommand{\totder}[2]{ {D{#1} \over D{#2} } }%
\newcommand{\ulvec}[1]{\underline{#1}}
%\def\well#1{\mathstrut #1}%
%\def\I{\, \mathord{\sl i}\, }%
% ---------------------------------------------------------
% the classic sets
% ---------------------------------------------------------
\newcommand{\rea}{\mathbb{R}}%
\newcommand{\com}{\mathbb{C}}%
\newcommand{\reb}{\mathbb{R}^2}%
\newcommand{\ren}{\mathbb{R}^n}
% ---------------------------------------------------------
% math formulas, displaystyle, in LR mode
% ---------------------------------------------------------
\newcommand{\mathbox}[1]{%
\mbox{$\displaystyle #1$}%
}
\newcommand{\tavg}[1]{\left\langle #1 \right\rangle}
\newcommand{\tavh}[1]{\langle #1 \rangle}
% -----------------------------------------------------------------------------
% e e i
% -----------------------------------------------------------------------------
\newcommand{\me}{\mathrm{e}}
\newcommand{\mi}{\mathrm{i}}
\newcommand{\md}{\mathrm{d}}
%% \newcommand{\md}{\mathord{\textsl{d}}}
%% \newcommand{\mi}{\mathord{\textsl{i}}\,}
%% \newcommand{\me}{\mathord{\textsl{e}}}

% ---------------------------------------------------------------
% minha redefini��o local de vetor
% ---------------------------------------------------------------
%\newcommand{\vet}[1]{\mathbf{#1}}
\newcommand{\bdot}{\boldsymbol{\cdot}}
\newcommand{\vet}[1]{\boldsymbol{#1}}
\newcommand{\vut}[1]{\ensuremath{\mathbf{#1}}}
\newcommand{\vem}[1]{\underset{\sim}{#1}}
\newcommand{\ven}[1]{\underset{\approx}{#1}}
\newcommand{\veg}[1]{\mathord{\mbox{\boldmath $#1$}}}
%\newcommand{\mat}[1]{\mathord{[\mbox{\boldmath $#1$}]}}
%\newcommand{\met}[1]{\mathord{\mbox{\boldmath $[#1]$}}}
\newcommand{\met}[1]{\mbox{$\boldsymbol{[}$} \boldsymbol{#1} \mbox{$\boldsymbol{]}$}}
\newcommand{\mat}[1]{\mbox{$\boldsymbol{[}$} #1 \mbox{$\boldsymbol{]}$}}
\newcommand{\mut}[1]{\ensuremath{\mathbf{[#1]}}}
% --------------------------------------------------------------------
% mathptmx requires a special treatment
% --------------------------------------------------------------------
%\newcommand{\mut}[1]{\mathord{\mbox{       $\bm{[#1]}$}}}
% --------------------------------------------------------------------
% 
% --------------------------------------------------------------------
\newcommand{\bigmat}[1]{\mbox{$\boldsymbol{\big[}$} #1 \mbox{$\boldsymbol{\big]}$}}
\newcommand{\Bigmat}[1]{\mbox{$\boldsymbol{\Big[}$} #1 \mbox{$\boldsymbol{\Big]}$}}
\newcommand{\biggmat}[1]{\mbox{$\boldsymbol{\bigg[}$} #1 \mbox{$\boldsymbol{\bigg]}$}}
\newcommand{\Biggmat}[1]{\mbox{$\boldsymbol{\Bigg[}$} #1 \mbox{$\boldsymbol{\Bigg]}$}}
\newcommand{\ten}[1]{\mathbb{#1}}
\newcommand{\mlin}[1]{[#1]^{\mathsf{\small T}}}
%\newcommand{\tr} {{\mbox{\tiny$\mathsf{T}$}}}
\newcommand{\tr} {{\mbox{\tiny$\boldsymbol{\top}$}}}
\newcommand{\adj}{{\mbox{\tiny$\dagger$}}}
\newcommand{\tuplet}[1]{({#1}_1,{#1}_2,{#1}_3)}
\newcommand{\tuplen}[1]{({#1}_1,{#1}_2,\ldots,{#1}_n)}
\newcommand{\jac}[2]{\frac{\partial\tuplen{#1}}{\partial\tuplen{#2}}}
% -----------------------------------------------------------------------------
% latitude, longitude
% -----------------------------------------------------------------------------
\newcommand{\latmin}[3]{#1\textdegree\,#2\textquotesingle\,#3}
\newcommand{\lonmin}[3]{#1\textdegree\,#2\textquotesingle\,#3}
% -----------------------------------------------------------------------------
% � fun��o de/ n�o � fun��o de
% -----------------------------------------------------------------------------
%\newcommand{\isfunc}[1]{\mbox{=\textflorin}(#1)}
\newcommand{\isfunc}[1]{:=\text{f{\kern-0.1em}f}(#1)}
\newcommand{\isnotfunc}[1]{:\ne\text{f{\kern-0.1em}f}(#1)}
\newcommand{\isconst}[1]{:=\text{\textcent}}
\newcommand{\inner}[2]{\langle {#1},{#2} \rangle}
% ---------------------------------------------------------------------------
% macros to produce uppercase greek letters
% ---------------------------------------------------------------------------
\newcommand{\Alpha}{\mathord{\mathcal{A}}}
\newcommand{\Beta}{\mathord{\mathcal{B}}}
\newcommand{\Epsilon}{\mathord{\mathscr{E}}}
\newcommand{\Zeta}{\mathord{\mathcal{Z}}}
\newcommand{\Eta}{\mathord{\mathcal{H}}}
\newcommand{\Iota}{\mathord{\mathcal{I}}}
\newcommand{\Kappa}{\mathord{\mathcal{K}}}
\newcommand{\Mu}{\mathord{\mathcal{M}}}
\newcommand{\Nu}{\mathord{\mathcal{N}}}
% \newcommand{\omicron}{\mathord{o}}
\newcommand{\Omicron}{\mathord{\mathcal{O}}}
%\newcommand{\Rho}{\mathord{\large\mbox{$\wp$}}}
%\newcommand{\Rho}{\mathord{\mbox{\large{$\wp\!$}}}}
%\newcommand{\Rho}{{\mbox{\large{$\wp{\kern -.1em}$}}}}
% ---------------------------------------------------------------------------
% I don't know how to do this with \newcommand
% ---------------------------------------------------------------------------
\def\Rho{%
\mathchoice%
      {\mbox{\large{$\wp\kern-0.05em$}}}
      {\mbox{\large{$\wp$}}}
      {\scriptstyle\wp}
      {\scriptscriptstyle\wp}
}



% \newcommand{\Rho}{\mathord{\mathcal{P}}}
\newcommand{\Tau}{\mathord{\mathcal{T}}}
\newcommand{\Chi}{\mathord{\mathcal{X}}}
%%\newcommand{\tet}{\mathord{\mbox{\footnotesize$\mathpzc{T}$}}}
\newcommand{\tet}{\mathord{\mbox{\footnotesize$\EuScript{T}$}}}
%%\newcommand{\vsp}{\mathord{\mbox{$\mathscr{V}$}}}
\newcommand{\vsp}{\mathord{\mbox{\small$\mathpzc{V}$}}}
\newcommand{\Ein}{\EuScript{U}}
\newcommand{\Hel}{\EuScript{F}}
\newcommand{\Gib}{\EuScript{G}}

%% -----------------------------------------------------------------------------
%% neat trick!
%% -----------------------------------------------------------------------------
%%\newcommand{\sfbTheta}{\text{{\biolinumGlyph{uni0398}}}}
%%\newcommand{\varRho}{\text{{\libertineGlyph{uni01A4}}}}



%% \newcommand{\Alpha}{\mathord{\mbox{\zapf{A}}}}
%% \newcommand{\Beta}{\mathord{\mbox{\zapf{B}}}}
%% \newcommand{\Epsilon}{\mathord{\mbox{\zapf{E}}}}
%% \newcommand{\Zeta}{\mathord{\mbox{\zapf{Z}}}}
%% \newcommand{\Eta}{\mathord{\mbox{\zapf{H}}}}
%% \newcommand{\Iota}{\mathord{\mbox{\zapf{I}}}}
%% \newcommand{\Kappa}{\mathord{\mbox{\zapf{K}}}}
%% \newcommand{\Mu}{\mathord{\mbox{\zapf{M}}}}
%% \newcommand{\Nu}{\mathord{\mbox{\zapf{N}}}}
%% \newcommand{\omicron}{\mathord{\mbox{\zapf{o}}}}
%% \newcommand{\Omicron}{\mathord{\mbox{\zapf{O}}}}
%% %%\newcommand{\Rho}{\mathord{\mbox{\zapf{P}\,}}}
%% \newcommand{\Rho}{\mbox{\Large$\wp$}}
%% \newcommand{\Tau}{\mathord{\mbox{\zapf{T}}}}
%% \newcommand{\Chi}{\mathord{\mbox{\zapf{X}}}}


% ---------------------------------------------------------------------------
% macros to produce punctuation inside math mode
% ---------------------------------------------------------------------------
%\newcommand{\MathPeriod}{\;{\rm .}}
%\newcommand{\MathComma}{\;{\rm ,}}
%\newcommand{\MathSemiColon}{\;{\rm ;}}
%\newcommand{\MathColon}{\;{\rm :}}
% --------------------------------------------------------------------------
% EqnList provides an environment with a list of equations with numbers 
% like 1-a, 1-b, ...  It works ALMOST like the eqnarray 
% environment,
% but preceding each equation you MUST type \EqnItem
% --------------------------------------------------------------------------
%\newcommand{\EqnItem}{%
%   \addtocounter{ItemNumber}{1}%
%   \addtocounter{equation}{-1}%
%}%
%\newenvironment{EqnList}{%
%   \renewcommand{\theequation}%
%   {\arabic{equation}-\alph{ItemNumber}}%
%   \addtocounter{equation}{1}%
%   \begin{eqnarray}%
%}%
%{%
%   \end{eqnarray}%
%   %\addtocounter{equation}{-1}%
%   \setcounter{ItemNumber}{0}%
%   \renewcommand{\theequation}{\arabic{equation}}%
%}%
\newcommand{\mathsci}[1]{\mathord{\mbox{\small\scfamily\itshape #1}}}
\newlength{\equal}\settowidth{\equal}{$\:=\:$}


















